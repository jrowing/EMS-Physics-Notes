\section*{Appendix B. List of Formulae}
\addcontentsline{toc}{part}{Appendix B. List of Formula}
\rhead{ \fancyplain{}{LIST OF FORMULAE} }

\vspace*{-2em}
%{\setlength{\baselineskip}%
	%	{4\baselineskip}
	{\setstretch{2.4}
		{
			\scriptsize

\begin{longtable}{p{.21\textwidth} p{.79\textwidth}}

$p=\sqrt{p_x^2 + p_y^2}$ & magnitude of a vector in terms of its components \\

$p_x = p\cos\theta, \, p_y = p\sin\theta$ & $x$- and $y$-components of a vector \\

$v = \frac{\Delta s}{\Delta t}$ & defining equation for velocity \\

$a = \frac{\Delta v}{\Delta t}$ & defining equation for acceleration \\

$s=vt$ & displacement-time relation for motion with uniform velocity in a straight line \\

$v=u+at$ & velocity-time relation for motion with constant acceleration \\

$s=ut+\frac{1}{2}at^2$ & displacement-time relation for motion with constant acceleration \\

$s=\frac{1}{2}(u+v)t$ & displacement-time relation for motion with constant acceleration \\

$v^2 - u^2 = 2as$ & displacement-velocity relation for motion with constant acceleration \\

$v_x = u\cos\theta$ & horizontal velocity for a projectile with initial velocity $u$ at angle $\theta$ above horizontal \\

$x = u\cos\theta \cdot t$ & horizontal displacement for a projectile with initial velocity $u$ at angle $\theta$ above horizontal \\

$v_y = u \sin\theta  - gt$ & vertical velocity for a projectile with initial velocity $u$ at angle $\theta$ above horizontal \\

$y = u \sin\theta \cdot t - \frac{1}{2}gt^2$ & height-time relation for a projectile with initial velocity $u$ at angle $\theta$ above horizontal \\

$p=mv$ & defining equation for momentum \\

$F = \frac{\Delta p}{\Delta t}$ & defining equation for force \\

$m_1u_1 + m_2 u_2 = m_1 v_1 + m_2 v_2$ & two-body collision in one dimension in absence of external forces \\

$|u_1-u_2| = |v_1-v_2|$ & relative speed unchanged during a perfectly elastic two-body collision \\

$F_\text{net} = ma$ & Newton's second law \\

$W = mg$ & weight of object near surface of the earth \\

$mg-f=ma$ & equation for motion of an object falling through air or any viscous fluid \\

$W_\parallel = mg\sin\theta$ & component of weight down an inclined slope at angle $\theta$ to horizontal\\

$W_\perp =mg\cos\theta$ & component of weight perpendicular to an inclined slope at angle $\theta$ to horizontal\\

$W = Fs(\cos\theta)$ & work done by a force (at an angle of $\theta$ to displacement moved out) \\

$P = \frac{\Delta W}{\Delta t}$ & defining equation for power \\

$P = F v$ & instantaneous power of a force on a body travelling at velocity $v$ \\

$E_k = \frac{1}{2}mv^2$ & kinetic energy associated with a moving body \\

$E_p = mgh$ & potential energy possessed by a body due to its position in a gravitational field \\

$T = kx$ & Hooke's law, tension in ideal springs is proportional to extension/compression $x$\\

$E_p = \frac{1}{2}kx^2$ & elastic potential energy in a spring with extension/compression $x$ \\

$\sigma = \frac{F}{A}$ & tensile stress on a material \\

$\epsilon = \frac{x}{L}$ & tensile strain of a deformed body \\

$Y = \frac{\sigma}{\epsilon}$ & definition for Young's modulus of a material (CAIE notation: $E$ for Young's modulus) \\

$p=\frac{F}{A}$ & pressure of a force acting or a surface \\

$p=\rho g h$ & pressure due to self-weight of a fluid at a depth of $h$ beneath the surface \\

$F = \rho g V_\text{disp}$ & Archimedes' principle on the upthrust on an object immersed in a fluid \\

$\tau = F d_\perp$ & moment of a force about a fixed pivot \\

% $E = \frac{F}{q}$ & defining equation for electric field strength \\

% $E = \frac{V}{d}$ & field strength between two parallel oppositely-charged metal plates separated by $d$ \\

$I = \frac{\Delta Q}{\Delta t}$ & defining equation for electric current \\

$I = nAqv$ & electric currents in terms of microscopic motion of charge carriers \\

$V = \frac{\Delta W}{\Delta Q}$ & defining equation for electric potential difference (p.d.) \\

$\sum I_\text{in} = \sum I_\text{out}$ & Kirchhoff's first law of current electricity \\

$V=IR$ & Ohm's law, relating p.d. across and current through a resistor \\

$\sum \mathcal{E}_i = \sum I_i R_i$ & Kirchhoff's second law of current electricity \\

$R=R_1+R_2+R_3+\cdots$ & combined resistance of several resistors connected in series \\

$\frac{1}{R} = \frac{1}{R_1} + \frac{1}{R_2} + \frac{1}{R_3} +\cdots$ & combined resistance of several resistors connected in parallel \\

$R = \frac{\rho L}{A}$ & dependence of resistance on length, cross-section and material of conductor \\

$P=IV \text{ or } I^2R \text{ or } \frac{V^2}{R}$ & power dissipated in a resistor \\

$V = \mathcal{E} - Ir$ & terminal p.d. across a power supply when there exist lost volts to internal resistance $r$\\

$f=\frac{1}{T}$ & relating frequency and period of wave motion \\

$v=\frac{\lambda}{T} = \lambda f$ & speed at which wave propagates through space \\

$f = \frac{f_0}{1\mp v_s/v}$ & observed frequency when wave source is moving towards ($-$)/away from ($+$) at speed $v_s$\\

$ I = \frac{P}{A}$ & wave intensity defined as power per unit area \\

$ I \propto A^2 $ & wave intensity proportional to the square of wave amplitude \\

$ I \propto \frac{1}{d^2}$ & decrease of wave intensity due to a point source with distance \\

$ I = I_0 \cos^2\theta $ & Malus's law: intensity of a plane-polarized light passing through an analyzer at angle $\theta$ \\

$\Delta \phi = 0, 2\pi, 4\pi, 6\pi, \cdots$ & phase difference condition for constructive interference between two waves \\

$\Delta L = 0, \lambda, 2\lambda, 3\lambda, \cdots$ & path difference condition for constructive interference between two waves \\

$\Delta \phi = \pi, 3\pi, 5\pi, \cdots$ & phase difference condition for destructive interference between two waves \\

$\Delta L = \frac{1}{2}\lambda, \frac{3}{2}\lambda, \frac{5}{2}\lambda, \cdots$ & path difference condition for destructive interference between two waves \\

$x = \frac{\lambda D}{d}$ & fringe separation for a double-slit interference pattern \\

$d\sin\theta = n\lambda$ & diffraction grating equation \\

$n_\text{max} = \left\lceil \frac{d}{\lambda} \right\rceil$ & highest order produced as light is sent through a diffraction grating\\

$N = 2n_\text{max} + 1$ & number of maxima produced as light is sent through a diffraction grating\\

$A=Z+N$ & nucleon (mass) number = proton (charge) number + neutron number \\

$^A_Z X \longrightarrow ^{A-4}_{Z-2} Y + ^4_2 \alpha$ & $\alpha$-decay from an unstable nucleus \\

$^A_Z X \longrightarrow ^{\phantom{+1}A}_{Z+1} Y + ^{\phantom{-}0}_{-1} \beta + ^0_0 \bar{\nu}_e$ & $\beta^-$-decay from an unstable nucleus \\

$^1_1 p \longrightarrow ^1_0 n + ^{\phantom{-}0}_{-1} e^- + ^0_0 \bar{\nu}_e$ & $\beta^-$-decay as a neutron changes into a proton\\

$u \longrightarrow d + ^{\phantom{-}0}_{-1} e^- + ^0_0 \bar{\nu}_e$ & $\beta^-$-decay at the fundamental level: an up quark changes into a down quark\\

$^A_Z X \longrightarrow ^{\phantom{+1}A}_{Z-1} Y + ^{\phantom{+}0}_{+1} \beta + ^0_0 \nu_e$ & $\beta^+$-decay from an unstable nucleus \\

$^1_0 n \longrightarrow ^1_1 p + ^{\phantom{+}0}_{+1} e^+ + ^0_0 \nu_e$ & $\beta^+$-decay as a proton changes into a neutron\\

$d \longrightarrow u + ^{\phantom{-}0}_{+1} e^+ + ^0_0 \nu_e$ & $\beta^+$-decay at the fundamental level: a down quark changes into an up quark\\

% $\Delta E = \Delta m c^2$ & mass-energy equation \\

\end{longtable} 

}\par}
