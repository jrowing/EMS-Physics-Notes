\chapter{Cosmology}

\section{Stephan's law}
An objects luminosity depends on two factors:
Its surface temperature
Its surface area
The relationship between these is known as Stefan's Law or the Stefan-Boltzmann Law, which states:
The total energy emitted by a black body per unit area per second is proportional to the fourth power of the absolute temperature of the body

So Stefan's Law shows that the luminosity of a star is directly proportional:
\begin{itemize}
    \item To its radius $L \propto r^2$
    \item To its surface area $L \propto 4 \pi r^2$
    \item To its surface absolute temperature $L \propto T^4$
\end{itemize}


Stefan's Law equation is given by:
\begin{equation*}
    L = 4 \pi r2\sigma T^{4}
\end{equation*}

Where:
L = luminosity of the star (W)\\
r = radius of the star\\
$\sigma$ = the Stefan-Boltzmann constant\\
T = surface temperature of the star (K)\\
The surface area of a star (or other spherical object) can be calculated using: $A = 4\pi r^2$\\
Where r = radius of the star

\example{The surface temperature of Proxima Centuri, the nearest star to Earth, is $3000$K and its luminosity is $6.506 \times 10^{23} $W.

Calculate the radius of Proxima Centuri in kilometres and show your working clearly.}
\begin{soln}
List the known quantities: 

Surface temperature, $T = 3000 K$\\
Luminosity, $L = 6.506 \times 10^{23} $W\\
Stefan's constant, $\sigma = 5.67 \times 10^{-8} W m^{-2} K^{-4}$\\
Write down Stefan's Law \& sub in values\\
The radius of Proximal Centuri is 106 200 km (4 s.f.)
\end{soln}
\section{Wein's law}

\subsection{Black Body Radiator}
An ideal black body radiator is one that absorbs and emits all wavelengths.\\
A black body is a theoretical object, however, stars are the best approximation there is.\\
The radiation emitted from a black body has a characteristic spectrum that is determined by the temperature alone.
% CUSTOM COLORS
% See https://tikz.net/blackbody_color/
\definecolor{1000K}{rgb}{1,0.0337,0}
\definecolor{2000K}{rgb}{1,0.2647,0.0033}
\definecolor{3000K}{rgb}{1,0.4870,0.1411}
\definecolor{4000K}{rgb}{1,0.6636,0.3583}
\definecolor{5000K}{rgb}{1,0.7992,0.6045}
\definecolor{6000K}{rgb}{1,0.9019,0.8473}
\definecolor{8000K}{rgb}{0.7874,0.8187,1}
\definecolor{10000K}{rgb}{0.6268,0.7039,1}
\pgfdeclareverticalshading{rainbow}{100bp}{
  color(0bp)=(red); color(25bp)=(red); color(35bp)=(yellow);
  color(45bp)=(green); color(55bp)=(cyan); color(65bp)=(blue);
  color(75bp)=(violet); color(100bp)=(violet)
}
\colorlet{myred}{red!70!black}
\colorlet{mygreen}{green!70!black}
\colorlet{mydarkgreen}{green!55!black}
\pgfmathdeclarefunction{planck}{2}{%
  \pgfmathparse{1.191042972e26/(#1^5)/(exp(0.01439e9/(#1*#2))-1)}%
}
\pgfmathdeclarefunction{rayleighjeans}{2}{%
  \pgfmathparse{8.278160269e18*#2/(#1^4)}%
}
\pgfmathdeclarefunction{wien}{2}{%
  \pgfmathparse{1.191042972e26/(#1^5)*exp(-0.01439e9/(#1*#2))}%
}
\pgfmathdeclarefunction{lampeak}{1}{% % Wien's displacement law
  \pgfmathparse{2.898e6/#1}%
}
\begin{figure*}

\begin{tikzpicture}[scale=1.5]
  \def\N{60}
  \def\xmax{3100}
  \def\ymax{1.36e10}
  \def\tick#1#2{\draw[thick] (#1+.01*\ymax) -- (#1-.01*\ymax) node[below=-.5pt,scale=0.75] {#2};}
  \begin{axis}[
      every axis plot/.style={
        mark=none,samples=\N,domain=5:\xmax,smooth},
      xmin=(-.05*\xmax), xmax=(1.05*\xmax),
      ymin=(-.08*\ymax), ymax=(1.08*\ymax),
      restrict y to domain=0:\ymax,
      axis lines=middle,
      axis line style=thick,
      %enlargelimits=upper, % extend the axes a bit to the right and top
      tick style={black,thick},
      ticklabel style={scale=0.8},
      %xtick style={draw=none},xticklabels=none,
      max space between ticks=26,
      xlabel={Wavelength $\lambda$ [nm]},
      ylabel={Power $P$ [kW/sr\,m$^2$\,nm]},
      xlabel style={at={(rel axis cs:0.5,0)},below=-1pt,font=\small},
      ylabel style={at={(rel axis cs:-0.11,0.5)},rotate=90},
      width=9cm, height=7cm,
      %clip=false
      tick scale binop=\times,
      every y tick scale label/.style={at={(rel axis cs:0,1)},anchor=south}]
    ]
    
    % RAINBOW
    \shade[shading=rainbow,shading angle=90,opacity=0.5] (380,0) rectangle (740,\ymax);
    \node[above=-1pt,scale=0.8] at (200,\ymax) {\strut UV}; % 10 - 400 nm
    \node[above=-1pt,scale=0.8] at (570,\ymax) {\strut optical}; % 380 - 740 nm
    \node[above=-1pt,scale=0.8] at (920,\ymax) {\strut IR}; % 740 - 1050 nm
    
    % PLANCK
    \addplot[very thick,red]    {planck(x,3000)};
    \addplot[very thick,orange] {planck(x,4000)};
    \addplot[very thick,samples=3*\N,blue] {planck(x,5000)};
    %\addplot[dashed,thick,red,domain=1000:4000]    {rayleighjeans(x,3000)};
    %\addplot[dashed,thick,orange,domain=1000:4000] {rayleighjeans(x,4000)};
    %\addplot[dashed,thick,blue,domain=1000:4000]   {rayleighjeans(x,5000)};
    %\addplot[dashed,thick,red,domain=1000:4000]    {wien(x,3000)};
    %\addplot[dashed,thick,orange,domain=1000:4000] {wien(x,4000)};
    %\addplot[dashed,thick,blue,domain=1000:4000]   {wien(x,5000)};
    
    %% MAXIMUM (Wien's displacement law)
    \addplot[mygreen,thin,variable=T,domain=2500:6000]({lampeak(T)},{planck(lampeak(T),T)});
    
    % LABELS
    \node[above=0pt,scale=0.75,red] at (1150,{planck(1150,3000)}) {3000K};
    \node[above right=-1pt,scale=0.75,orange!80!black] at (740,{planck(740,4000)}){4000K};
    \node[above right=-1pt,scale=0.75,blue] at (800,{planck(800,5000)}) {5000K};
    
    %% TICKS
    %\tick{500,0}{500}
    %\tick{1000,0}{1000}
    %\tick{1500,0}{1500}
    %\tick{2000,0}{2000}
    %\tick{2500,0}{2500}
    %\tick{3000,0}{3000}
  \end{axis}
\end{tikzpicture}
\end{figure*}

The intensity-wavelength graph shows how thermodynamic temperature links to the peak wavelength for three different bodies.

\subsection {Wien's Displacement Law}
Wien’s displacement law relates the observed wavelength of light from an object to its surface temperature, it states:
The black body radiation curve for different temperatures peaks at a wavelength that is inversely proportional to the temperature

This relation can be written as:

\begin{equation*}
    \lambda_{\mathrm{max}}=\frac{W}{T}
\end{equation*}

Where:\\
$\lambda_{\mathrm{max}}$ = the maximum wavelength emitted by an object at the peak intensity (m) \\
T = the surface temperature of an object (K) \\
Wien's constant is $W=2.9\times 10^{-3} mK $

This equation shows that the higher the temperature of a body, the shorter the wavelength at the peak intensity. Hotter objects tend to be white or blue, and cooler objects tend to be red or yellow.//


\example{
The spectrum of the star Rigel in the constellation of Orion peaks at a wavelength of 263 nm, while the spectrum of the star Betelgeuse peaks at a wavelength of 828 nm.
Determine which of these two stars, Betelgeuse or Rigel, is cooler.}
\begin{soln}
List the known quantities
Maximum emission wavelength of Rigel = 263 nm = 263 × 10-9 m
Maximum emission wavelength of Betelgeuse = 828 × 10-9 m

$$T = \frac{2.9e-3}{\lambda_{max}}$$
Calculate the surface temperature of each star:
$$\text{Rigel: }T=\frac{2.9e{-3}}{263e{-9}}=11026 = 11 000K$$

$$\text{Betelgeuse: }T=\frac{2.9e{-3}}{828e{-9}}=3502 = 3500K$$

Step 4: Write a concluding sentence

Betelgeuse has a surface temperature of 3500 K, therefore, it is much cooler than Rigel
\end{soln}
